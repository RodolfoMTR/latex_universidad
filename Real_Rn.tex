\documentclass{article}
\usepackage[14pt]{extsizes}%modified font size
\usepackage{amsfonts,amsmath,amssymb,amsthm}
\usepackage[utf8]{inputenc}
\usepackage[spanish]{babel}
\newcommand{\R}{\mathbb R}
\newcommand{\N}{\mathbb N}
\newcommand{\Z}{\mathbb Z}
\newcommand{\Q}{\mathbb Q}
\newcommand{\I}{\mathbb I}
\title{Análisis en $\R^n$}
\author{Rodolfo Turpo R.}
\date{\today}
\begin{document}
\maketitle
\section{Espacios Métricos}
\noindent\textbf{punto interior}

sea $E\subset X$ y $x_0\in X$

\[x_0\text{ es punto interior de $E$ si }\exists r>0/B(x_0,r)\subset E\]

Si $x_0$ es un punto interior de $E$, entonces $x_0$ en realidad debe ser un elemento de $E$

El conjunto de todos los puntos interiores de $E$ se denota por $E'$

\noindent\textbf{punto adherente}

Sea $E\subset X$ y $x_0\in X$, entonces

\[x_0 \text{ es punto adherente de }E\,\Leftrightarrow \forall r>0\, B(x_0,r)\cap E\neq \emptyset\]

El conjunto de puntos adherentes de X se llama \textbf{cerradura} de X y se denota con $\overline{X}$
\[x_0 \in \overline{X} \text{ si }\forall r>0\, B(x_0,r)\cap E\neq \emptyset\]

\noindent\textbf{punto de acumulación}

$p$ es punto de acumulación de X, si y sólo si, $\exists (x_n)_{n\in m}$ de X de modo que $x_n\neq 0\forall n\in N$ tal que $ x_n \rightarrow p$

\noindent\textbf{Conjunto Cerrado}

Un conjunto $X$ es cerrado si coincide con su cerradura $(\overline{X})$

$X$ es cerrado si $\forall p\in \overline{X}\,\exists\varepsilon>0/B(p,\varepsilon)\cap X\neq 0$

\noindent\textbf{Conjunto Abierto}

$A\subset\R^n$ es abierto si $\forall p\in A\,\exists\varepsilon>0/B(p,\varepsilon)\subset A$

\noindent\textbf{Conjunto Compacto}

$x\subset\R^n$ es un conjunto compacto si es cerrado y acotado.

\noindent\textbf{Conjunto conexo}

Un conjunto es conexo cuando no admite separación, osea, $X\subset\R^n$ es conexo cuando no se puede expresar como la unión disjunta de dos conjuntos abiertos de $X$
%--------------------------------------------------------------

\section{Diferenciación en $\R^n$}

\subsection{Derivada Parcial}

Sea $U\subset\R^n\rightarrow\R$ una función definida en un abierto $U\subset\R^n$.Para todo $v=(\alpha_1,...,\alpha_n)\in \R^n$ con $a+v\in U$, la i-ésima derivada parcial de $f$ en $a\in U$  es el limite
\[\frac{\partial f}{\partial x_i}(a)=\lim_{t\rightarrow 0}\frac{f(a+te_i)-f(a)}{t}\] cuando tal limite existe.

\subsection{Derivada direccional}

Sea $U\subset\R^n\rightarrow\R$ una función definida en un abierto $U\subset\R^n$. la derivada direccional de $f$ en $a\in U$ en la dirección $v\in R^n$ es definida como
\[\frac{\partial f}{\partial v}(a)=\lim_{t\rightarrow0}\frac{f(a+tv)-f(a)}{t}\]
Siempre que este limite exista.

\subsection{Funciones diferenciables}

Sea $U\subset\R^n\rightarrow\R$ una función definida en un abierto $U\subset\R^n$. Decimos que $f$ es \textbf{Diferenciable} en $a\in U$ si para todo $v=(\alpha_1,...,\alpha_n)\in \R^n$ con $a+v\in U$ se tiene que existen constantes $A_1,...,A_n$ de modo que 
\[f(a+v)-f(a)=A_1\alpha_1+...+A_n\alpha_n+r(v)\text{ donde } \lim_{v\rightarrow0} \frac{r(v)}{|v|}=0\]

\textbf{Observaciones}
\begin{itemize}
    \item si $A=(A_1,...,A_n)$, entonces $A_1\alpha_1+...+A_n\alpha_n=\langle A,v\rangle$
    \item sea: $\partial f_a=\left(\frac{\partial f}{x_1}(a),...,\frac{\partial f}{x_n}(a)\right)$, entonces
    \[f(a+v)-f(a)=\sum_{i=1}^n\frac{\partial f}{\partial x_i}(a)\alpha_i+r(v)/\lim_{v\rightarrow0} \frac{r(v)}{|v|}=0\]
    \item Si $f$ es diferenciable en $a$, entonces $f$ es continua
\end{itemize}
\noindent\textbf{Teorema del Valor Intermedio}

sea $f:X\rightarrow\R$ una función continua, definida en un conjunto conexo $X\subset\R^n$, Si existen $a,b\in X$ y $d\in\R$ tales que $f(a)<d<f(b)$ entonces existe un $c\in X$ tal que $f(c)=d$.

\noindent\textbf{Teorema del Valor Medio}


Sean $U\subset \mathbb R^n$ un abierto, $a\in U$ y $v\in R^n$ y $f:U\subset\mathbb R^n\rightarrow \mathbb R$ una función de modo que $f|_{[a,a+v]}$ es continua y tal que $\frac{\partial f}{\partial v}(x)$ existe un $\theta\in (0,1)$ que cumple

\[f(a+v)-f(a)=\frac{\partial f}{\partial v}(a+\theta v)\]

\noindent\textbf{Derivada de una función}

Sea $U\subset\R^n\rightarrow\R$ una función definida en un abierto $U\subset\R^n$. Si $f$ es diferenciable en $a\in U$, la derivada de $f$ en $a$ es una transformación lineal
\[\partial f_a:\R^n\rightarrow\R\]
definida como: 
\[\partial f_a v=\left\langle\left(\frac{\partial f}{y_1}(a),...,\frac{\partial f}{y_n}(a)\right),v\right\rangle\]
En otras palabras
\[\partial f_a v=\sum_{i=1}^n \frac{\partial f}{x_i}(a)\alpha_i\quad v=(\alpha_1,...,\alpha_n)\]

\noindent\textbf{Gradiente}

Sea $U\subset\R^n\rightarrow\R$ una función diferenciable definida en un abierto $U\subset\R^n$. El gradiente de $f$ en $a\in U$ se define como el vector "Grad $f(a)$" dado por
\[\langle grad\,f(a),v\rangle=\partial_{f_a} v\]
Para todo $v\in \R^n$. 
Tener en cuenta que el Gradiente es lo mismo que la derivada direccional

\textbf{Observaciones}

\[ grad\,f(a)=\left(\frac{\partial f}{\partial x_1}(a),...,\frac{\partial f}{\partial x_n}(a)\right)\]
para espacios euclidianos.

\noindent\textbf{Teorema de Schawrz}

Sea $f:U\subset R^n\rightarrow R$ dos veces diferenciable en el punto $c\in U\subset\R^n$. para cualquier $1\leq i,j\leq n$, se tiene
\[\frac{\partial^2 f}{\partial x_i\partial x_j}(c)=\frac{\partial^2 f}{\partial x_j\partial x_i}(c)\]

\end{document}