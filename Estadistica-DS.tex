\documentclass{article}
\usepackage[14pt]{extsizes}%modified font size
%--------------REFERENCIAS,CITAS
\usepackage{hyperref}
\hypersetup{
    colorlinks=true,
    linkcolor=blue,
    filecolor=magenta,      
    urlcolor=cyan,
    pdftitle={Overleaf Example},
    pdfpagemode=FullScreen,
    }

\urlstyle{same}
%--------------------
\usepackage{tikz}
\usepackage{amsfonts,amsmath,amssymb,amsthm}
\usepackage[utf8]{inputenc}
\usepackage[spanish]{babel}
\newcommand{\C}{\mathbb C}
\newcommand{\R}{\mathbb R}
\newcommand{\N}{\mathbb N}
\newcommand{\Z}{\mathbb Z}
\newcommand{\Q}{\mathbb Q}
\newcommand{\I}{\mathbb I}
\usepackage[left=2.5cm, right=2.5cm, top=2.5cm, bottom=2.5cm]{geometry}
\title{Ecuaciones Diferenciales Parciales\\Apuntes de Clase}
\author{Rodolfo Turpo Ramos}
\date{\Today}
%------------------DEFINICIONES,THEOREMAS,LEMAS
\newtheorem{definition}{Definición}
\renewcommand{\qedsymbol}{$\blacksquare$}
\newtheorem{theorem}{Teorema}
\newtheorem{lemma}{Lema}
\newtheorem{Acla}{Aclaración}
%-----------------------------------------
\begin{document}
De la media, se considera $\pm 3\sigma$, ahí se encuentra la mayoría de datos, exactamente 99.72\% de toda la distribución.

en términos del rango itercuartil IQR
Método de detección de outliers con IQR
min: $Q_1-1.5ISQ$
max: $Q_3+1.5IQR$
luego, este es el valor mínimo y máximo de lo que vamos a considerar datos normales, y todo lo que este fuera serán considerados datos anómalos.

que pasa cuando la dis. de datos no es simétrica, y es sesgada?
en este caso no aplica el método anterior, lo que se hace es generalizar el criterio anterior para desestimar los puntos anómalos. entonces en general\\
min: $Q_1-1.5f(IQR)$\\
max: $Q_2+1.5g(IQR)$\\
la desviación estandar es otra medida para calcular la dispersión de los datos de un conjunto cualquiera
\end{document}